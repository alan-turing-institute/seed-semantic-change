\documentclass{article}

\usepackage{amsmath}
\usepackage{bm}
\usepackage{caption}
\usepackage{subcaption}
\usepackage{enumerate}
\usepackage{float}
\usepackage{mathtools}
\usepackage[section]{placeins}
\usepackage{url}
\usepackage[ruled]{algorithm2e}
\usepackage{hyperref}
\usepackage{color}
\usepackage[a4paper, total={7in, 10in}]{geometry}
\usepackage{amssymb}
\usepackage{color}
\usepackage{enumitem,xcolor}
\usepackage[normalem]{ulem}


\title{Directions for a novel Bayesian meaning change model}
\author{Valerio Perrone}



\begin{document}
\maketitle 

\section{Use genre information}
\begin{itemize}
\item Simple approach: Lea's model applied to partitions of the data based on genres (IN PROGRESS)
\item Add genre-dependent covariates to Gaussian model for each meaning, i.e., 
\begin{align*}
\phi^t \mid \phi^{-t}, k^{\phi} \sim  N \left( \frac{\phi_{t-1} +  \phi_{t+1}}{2} + \sum_{j}  \beta_{j} \textbf{1}_j , k^{\phi} \right ),
\end{align*}
where $\beta_{j}$ represents the impact of genre $j$ on the meaning probability $\phi^t$ (note: this is a draft, the generative model is unclear yet). 
\item Use ideas from author-topic model, where authors correspond to genres in our case (see \url{https://mimno.infosci.cornell.edu/info6150/readings/398.pdf})
\end{itemize}


\section{Introduce change points}
\begin{itemize}
\item Simple approach for fixed change points at known locations: Lea's model applied to consecutive segments of the total time period. These decouples the dependencies between consecutive meaning probabilities $\phi$ and $\phi^{t+1}$.
\end{itemize}
\section{Allow for infinite meanings}
\begin{itemize}
\item Replace Gaussian time-evolution with dependent Dirichlet (see literature)
\item Extend to Hierarchical Dirichlet process
\end{itemize}
\section{More realistic time-evolution}
\begin{itemize}
\item Instead of Gaussian: evolution of normalized probabilities, values forced within 0 and 1, drift and more advanced correlation structure
\end{itemize}
\section{Add dependence across meanings}
\begin{itemize}
\item Use ideas from correlated topic models (see \url{http://people.ee.duke.edu/~lcarin/Blei2005CTM.pdf})
\end{itemize}


\end{document}\grid
\grid
\grid
\grid
